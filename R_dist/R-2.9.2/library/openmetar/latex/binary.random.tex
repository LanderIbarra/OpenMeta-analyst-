\HeaderA{binary.random}{}{binary.random}
\keyword{~kwd1}{binary.random}
\keyword{~kwd2}{binary.random}
\begin{Usage}
\begin{verbatim}
binary.random(binaryData, params)
\end{verbatim}
\end{Usage}
\begin{Arguments}
\begin{ldescription}
\item[\code{binaryData}] 
\item[\code{params}] 
\end{ldescription}
\end{Arguments}
\begin{Examples}
\begin{ExampleCode}
##---- Should be DIRECTLY executable !! ----
##-- ==>  Define data, use random,
##--    or do  help(data=index)  for the standard data sets.

## The function is currently defined as
function (binaryData, params) 
{
    if (!("BinaryData" %in% class(binaryData))) 
        stop("Binary data expected.")
    results <- NULL
    if (length(binaryData@g1O1) == 1 || length(binaryData@y) == 
        1) {
        res <- get.res.for.one.binary.study(binaryData, params)
        results <- list(summary = res)
    }
    else {
        if (length(binaryData@g1O1) > 0) {
            res <- rma.uni(ai = binaryData@g1O1, bi = binaryData@g1O2, 
                ci = binaryData@g2O1, di = binaryData@g2O2, slab = binaryData@studyNames, 
                method = params$rm.method, measure = params$measure, 
                level = params$conf.level, digits = params$digits)
        }
        else {
            res <- rma.uni(yi = binaryData@y, sei = binaryData@SE, 
                slab = binaryData@studyNames, method = params$rm.method, 
                level = params$conf.level, digits = params$digits)
        }
        degf <- res$k.yi - 1
        modelTitle <- paste("Binary Random-Effects Model (k = ", 
            res$k, ")", sep = "")
        summaryDisp <- createSummaryDisp(res, params, degf, modelTitle)
        summaryDisp
        getwd()
        forest_path <- "./r_tmp/forest.png"
        png(forest_path)
        forest.rma(res, digits = params$digits)
        dev.off()
        images <- c(`forest plot` = forest_path)
        plot_names <- c(`forest plot` = "forest_plot")
        results <- list(images = images, summary = summaryDisp, 
            plot_names = plot_names)
    }
    results
  }
\end{ExampleCode}
\end{Examples}

