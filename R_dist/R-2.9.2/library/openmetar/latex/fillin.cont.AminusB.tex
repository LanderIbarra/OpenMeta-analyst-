\HeaderA{fillin.cont.AminusB}{}{fillin.cont.AminusB}
\keyword{~kwd1}{fillin.cont.AminusB}
\keyword{~kwd2}{fillin.cont.AminusB}
\begin{Usage}
\begin{verbatim}
fillin.cont.AminusB(n.A = NA, mean.A = NA, sd.A = NA, se.A = NA, var.A = NA, low.A = NA, high.A = NA, pval.A = NA, n.B = NA, mean.B = NA, sd.B = NA, se.B = NA, var.B = NA, low.B = NA, high.B = NA, pval.B = NA, correlation = 0, alpha = 0.05)
\end{verbatim}
\end{Usage}
\begin{Arguments}
\begin{ldescription}
\item[\code{n.A}] 
\item[\code{mean.A}] 
\item[\code{sd.A}] 
\item[\code{se.A}] 
\item[\code{var.A}] 
\item[\code{low.A}] 
\item[\code{high.A}] 
\item[\code{pval.A}] 
\item[\code{n.B}] 
\item[\code{mean.B}] 
\item[\code{sd.B}] 
\item[\code{se.B}] 
\item[\code{var.B}] 
\item[\code{low.B}] 
\item[\code{high.B}] 
\item[\code{pval.B}] 
\item[\code{correlation}] 
\item[\code{alpha}] 
\end{ldescription}
\end{Arguments}
\begin{Examples}
\begin{ExampleCode}
##---- Should be DIRECTLY executable !! ----
##-- ==>  Define data, use random,
##--    or do  help(data=index)  for the standard data sets.

## The function is currently defined as
function (n.A = NA, mean.A = NA, sd.A = NA, se.A = NA, var.A = NA, 
    low.A = NA, high.A = NA, pval.A = NA, n.B = NA, mean.B = NA, 
    sd.B = NA, se.B = NA, var.B = NA, low.B = NA, high.B = NA, 
    pval.B = NA, correlation = 0, alpha = 0.05) 
{
    succeeded <- TRUE
    comment <- ""
    res <- list(succeeded = succeeded)
    n.diff <- NA
    mean.diff <- NA
    sd.diff <- NA
    se.diff <- NA
    var.diff <- NA
    low.diff <- NA
    high.diff <- NA
    pval.diff <- NA
    z <- abs(qnorm(alpha/2))
    input.vector.A <- c(n.A, mean.A, sd.A, se.A, var.A, low.A, 
        high.A, pval.A)
    input.vector.B <- c(n.B, mean.B, sd.B, se.B, var.B, low.B, 
        high.B, pval.B)
    input.pattern <- list(A = !(is.na(input.vector.A)), B = !(is.na(input.vector.B)))
    fillin.A <- fillin.cont.1spell(n.A, mean.A, sd.A, se.A, var.A, 
        low.A, high.A, pval.A, alpha = alpha)
    comment <- paste(comment, paste("A", fillin.A$comment, sep = ":"), 
        sep = "|")
    fillin.B <- fillin.cont.1spell(n.B, mean.B, sd.B, se.B, var.B, 
        low.B, high.B, pval.B, alpha = alpha)
    comment <- paste(comment, paste("B", fillin.B$comment, sep = ":"), 
        sep = "|")
    if (identical(c(fillin.A$succeeded, fillin.B$succeeded), 
        c(TRUE, TRUE))) {
        mean.diff <- fillin.A$output["mean"] - fillin.B$output["mean"]
        var.diff <- (fillin.A$output["se"])^2 + (fillin.B$output["se"])^2
        -2 * correlation * (fillin.A$output["se"]) * (fillin.B$output["se"])
        se.diff <- sqrt(var.diff)
        low.diff <- mean.diff - z * se.diff
        high.diff <- mean.diff + z * se.diff
        pval.diff <- 2 * pnorm(-abs(mean.diff/se.diff))
        n.diff <- try(min(fillin.A$output["n"], fillin.B$output["n"]), 
            silent = TRUE)
        sd.diff <- try(var.diff * (n.diff - 1), silent = TRUE)
    }
    else {
        succeeded <- FALSE
    }
    output.vector <- c(n.diff, mean.diff, sd.diff, se.diff, var.diff, 
        low.diff, high.diff, pval.diff)
    output.names <- c("n", "mean", "sd", "se", "var", "low", 
        "high", "pval")
    names(output.vector) <- output.names
    res <- list(succeeded = succeeded, input.pattern = input.pattern, 
        output = output.vector, pre = fillin.A$output, post = fillin.B$output, 
        comment = comment, correlation = correlation)
    return(res)
  }
\end{ExampleCode}
\end{Examples}

