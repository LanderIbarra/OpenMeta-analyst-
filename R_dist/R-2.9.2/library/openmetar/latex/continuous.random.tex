\HeaderA{continuous.random}{}{continuous.random}
\keyword{~kwd1}{continuous.random}
\keyword{~kwd2}{continuous.random}
\begin{Usage}
\begin{verbatim}
continuous.random(contData, params)
\end{verbatim}
\end{Usage}
\begin{Arguments}
\begin{ldescription}
\item[\code{contData}] 
\item[\code{params}] 
\end{ldescription}
\end{Arguments}
\begin{Examples}
\begin{ExampleCode}
##---- Should be DIRECTLY executable !! ----
##-- ==>  Define data, use random,
##--    or do  help(data=index)  for the standard data sets.

## The function is currently defined as
function (contData, params) 
{
    if (!("ContinuousData" %in% class(contData))) 
        stop("Continuous data expected.")
    results <- NULL
    if (length(contData@studyNames) == 1) {
        res <- get.res.for.one.cont.study(contData, params)
        results <- list(summary = res)
    }
    else {
        if (length(contData@mean1) > 0) {
            res <- rma.uni(n1i = contData@N1, n2i = contData@N2, 
                m1i = contData@mean1, m2i = contData@mean2, sd1i = contData@sd1, 
                sd2i = contData@sd2, slab = contData@studyNames, 
                method = params$rm.method, measure = params$measure, 
                level = params$conf.level, digits = params$digits)
        }
        else {
            res <- rma.uni(yi = contData@y, sei = contData@SE, 
                slab = contData@studyNames, method = params$rm.method, 
                level = params$conf.level, digits = params$digits)
        }
        getwd()
        forest_path <- "./r_tmp/forest.png"
        png(forest_path)
        forest.rma(res, digits = params$digits)
        dev.off()
        images <- c(`forest plot` = forest_path)
        plot_names <- c(`forest plot` = "forest_plot")
        results <- list(images = images, summary = res, plot_names = plot_names)
    }
    results
  }
\end{ExampleCode}
\end{Examples}

