\HeaderA{cksum}{Compute Check Sum}{cksum}
\keyword{arith}{cksum}
\keyword{utilities}{cksum}
\begin{Description}\relax
return a cyclic redundancy checksum for each element in the argument.
\end{Description}
\begin{Usage}
\begin{verbatim}
cksum(a)
\end{verbatim}
\end{Usage}
\begin{Arguments}
\begin{ldescription}
\item[\code{a}] coerced to character vector
\end{ldescription}
\end{Arguments}
\begin{Details}\relax
NA's appearing in the argument are returned as NA's.

The default calculation is identical to that given in pseudo-code in the
ACM article (in the References).
\end{Details}
\begin{Value}
numeric vector of length \code{a}.
\end{Value}
\begin{Author}\relax
Steve Dutky \email{sdutky@terpalum.umd.edu}
\end{Author}
\begin{References}\relax
Fashioned from \code{cksum(1)} UNIX command line utility, i.e.,
\code{man cksum}.

Dilip V. Sarwate (1988).
Computation of Cyclic Redundancy Checks Via Table Lookup,
\emph{Communications of the ACM}, August 1988.
\bold{vol} 31, No.8  page 1008-1013
\end{References}
\begin{Examples}
\begin{ExampleCode}
   b <- "I would rather have a bottle in front of me than frontal lobotomy\n"
   cksum(b) == 1342168430 ## -> TRUE
\end{ExampleCode}
\end{Examples}

