\HeaderA{bitShiftL}{Bitwise Shift Operator (to the Left or Right)}{bitShiftL}
\aliasA{bitShiftR}{bitShiftL}{bitShiftR}
\keyword{arith}{bitShiftL}
\keyword{utilities}{bitShiftL}
\begin{Description}\relax
.......
\end{Description}
\begin{Usage}
\begin{verbatim}
bitShiftL(a, b)
bitShiftR(a, b)
\end{verbatim}
\end{Usage}
\begin{Arguments}
\begin{ldescription}
\item[\code{a}] numeric vector 
\item[\code{b}] integer vector 
\end{ldescription}
\end{Arguments}
\begin{Value}
numeric vector of the maximum length as \code{a} or \code{b} containing
the value of \code{a} shifted to the left or right by \code{b} bits.
NA is returned wherever the value of \code{a} or \code{b} is not finite,
or, wherever the magnitude of \code{a} is greater than or equal to 2**32.
\end{Value}
\begin{Examples}
\begin{ExampleCode}
  bitShiftR(-1,1) == 2147483647
  bitShiftL(2147483647,1) == 4294967294
  bitShiftL(-1,1) == 4294967294

\end{ExampleCode}
\end{Examples}

