\HeaderA{performance-class}{Class "performance"}{performance.Rdash.class}
\keyword{classes}{performance-class}
\begin{Description}\relax
Object to capture the result of a performance evaluation,
optionally collecting evaluations from several cross-validation or
bootstrapping runs.
\end{Description}
\begin{Details}\relax
A \code{performance} object can capture information from four
different evaluation scenarios:
\Itemize{
\item The behaviour of a cutoff-dependent performance measure across
the range of all cutoffs (e.g. \code{performance( predObj, 'acc' )} ). Here,
\code{x.values} contains the cutoffs, \code{y.values} the
corresponding values of the performance measure, and
\code{alpha.values} is empty.\\
\item The trade-off between two performance measures across the
range of all cutoffs (e.g. \code{performance( predObj,
      'tpr', 'fpr' )} ). In this case, the cutoffs are stored in
\code{alpha.values}, while \code{x.values} and \code{y.values}
contain the corresponding values of the two performance measures.\\
\item A performance measure that comes along with an obligatory
second axis (e.g. \code{performance( predObj, 'ecost' )} ). Here, the measure values are
stored in \code{y.values}, while the corresponding values of the
obligatory axis are stored in \code{x.values}, and \code{alpha.values}
is empty.\\
\item A performance measure whose value is just a scalar
(e.g. \code{performance( predObj, 'auc' )} ). The value is then stored in
\code{y.values}, while \code{x.values} and \code{alpha.values} are
empty.
}
\end{Details}
\begin{Section}{Objects from the Class}
Objects can be created by using the
\code{performance} function.
\end{Section}
\begin{Section}{Slots}
\describe{
\item[\code{x.name}:] Performance measure used for the x axis.
\item[\code{y.name}:] Performance measure used for the y axis.
\item[\code{alpha.name}:] Name of the unit that is used to create the parametrized
curve. Currently, curves can only be parametrized by cutoff, so
\code{alpha.name} is either \code{none} or \code{cutoff}.
\item[\code{x.values}:] A list in which each entry contains the x values
of the curve of this particular cross-validation run. x.values[[i]],
y.values[[i]], and alpha.values[[i]] correspond to each other.
\item[\code{y.values}:] A list in which each entry contains the y values
of the curve of this particular cross-validation run.
\item[\code{alpha.values}:] A list in which each entry contains the cutoff values
of the curve of this particular cross-validation run.
}
\end{Section}
\begin{Author}\relax
Tobias Sing \email{tobias.sing@mpi-sb.mpg.de},
Oliver Sander \email{osander@mpi-sb.mpg.de}
\end{Author}
\begin{References}\relax
A detailed list of references can be found on the ROCR
homepage at \url{http://rocr.bioinf.mpi-sb.mpg.de}.
\end{References}
\begin{SeeAlso}\relax
\code{\LinkA{prediction}{prediction}}, \code{\LinkA{performance}{performance}},
\code{\LinkA{prediction-class}{prediction.Rdash.class}}, \code{\LinkA{plot.performance}{plot.performance}}
\end{SeeAlso}

