\HeaderA{EndRule}{internal function}{EndRule}
\keyword{internal}{EndRule}
\begin{Description}\relax
internal function
\end{Description}
\begin{Usage}
\begin{verbatim}EndRule(x, y, k, 
             endrule=c("NA", "trim", "keep", "constant", "func"), Func, ...)\end{verbatim}
\end{Usage}
\begin{Arguments}
\begin{ldescription}
\item[\code{x}] numeric vector of length n
\item[\code{k}] width of moving window; must be an integer between one and n.
\item[\code{endrule}] character string indicating how the values at the beginning 
and the end, of the data, should be treated. Only first and last \code{k2} 
values at both ends are affected, where \code{k2} is the half-bandwidth 
\code{k2 = k \%/\% 2}.
\Itemize{
\item \code{"mad"} - applies the mad function to
smaller and smaller sections of the array. Equivalent to: 
\code{for(i in 1:k2) out[i]=mad(x[1:(i+k2)])}. 
\item \code{"trim"} - trim the ends; output array length is equal to 
\code{length(x)-2*k2 (out = out[(k2+1):(n-k2)])}. This option mimics 
output of \code{\LinkA{apply}{apply}} \code{(\LinkA{embed}{embed}(x,k),1,FUN)} and other 
related functions.
\item \code{"keep"} - fill the ends with numbers from \code{x} vector 
\code{(out[1:k2] = x[1:k2])}. This option makes more sense in case of 
smoothing functions, kept here for consistency.
\item \code{"constant"} - fill the ends with first and last calculated 
value in output array \code{(out[1:k2] = out[k2+1])}
\item \code{"NA"} - fill the ends with NA's \code{(out[1:k2] = NA)}
\item \code{"func"} - same as \code{"mad"} option except that implemented
in R for testing purposes. Avoid since it can be very slow for large windows.
}

\item[\code{y}] numeric vector of length n, which is partially filled output of 
one of the \code{run} functions. Function \code{EndRule} will fill the 
remaining beginning and end sections using method chosen by \code{endrule} 
argument.
\item[\code{Func}] Function name that \code{EndRule} will use in case of 
\code{endrule="func"}.
\item[\code{...}] Additional parameters to \code{Func} that \code{EndRule} will 
use in case of \code{endrule="func"}.
\end{ldescription}
\end{Arguments}
\begin{Value}
Returns a numeric vector of the same length as \code{x}. Only in case of 
\code{endrule="trim"}.the output will be shorter.
\end{Value}
\begin{Author}\relax
Jarek Tuszynski (SAIC) \email{jaroslaw.w.tuszynski@saic.com}
\end{Author}

