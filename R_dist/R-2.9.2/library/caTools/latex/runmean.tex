\HeaderA{runmean}{Mean of a Moving Window}{runmean}
\keyword{ts}{runmean}
\keyword{smooth}{runmean}
\keyword{array}{runmean}
\keyword{utilities}{runmean}
\begin{Description}\relax
Moving (aka running, rolling) Window Mean calculated over a vector
\end{Description}
\begin{Usage}
\begin{verbatim}
  runmean(x, k, alg=c("C", "R", "fast", "exact"),
         endrule=c("mean", "NA", "trim", "keep", "constant", "func"))
\end{verbatim}
\end{Usage}
\begin{Arguments}
\begin{ldescription}
\item[\code{x}] numeric vector of length n
\item[\code{k}] width of moving window; must be an integer between 1 and n 
\item[\code{alg}] an option to choose different algorithms
\Itemize{
\item \code{"C"} - a version is written in C. It can handle non-finite 
numbers like NaN's and Inf's (like \code{\LinkA{mean}{mean}(x, na.rm = TRUE)}). 
It works the fastest for \code{endrule="mean"}.
\item \code{"fast"} - second, even faster, C version. This algorithm
does not work with non-finite numbers. It also works the fastest for
\code{endrule} other than  \code{"mean"}.
\item \code{"R"} - much slower code written in R. Useful for 
debugging and as documentation.
\item \code{"exact"} - same as \code{"C"}, except that all additions 
areperformed using algorithm which tracks and corrects addition 
round-off errors
}

\item[\code{endrule}] character string indicating how the values at the beginning 
and the end, of the data, should be treated. Only first and last \code{k2} 
values at both ends are affected, where \code{k2} is the half-bandwidth 
\code{k2 = k \%/\% 2}.
\Itemize{
\item \code{"mean"} - applies the underlying function to smaller and 
smaller sections of the array. Equivalent to: 
\code{for(i in 1:k2) out[i]=mean(x[1:i])}. This option is implemented in 
C if \code{alg="C"}, otherwise is done in R.
\item \code{"trim"} - trim the ends; output array length is equal to 
\code{length(x)-2*k2 (out = out[(k2+1):(n-k2)])}. This option mimics 
output of \code{\LinkA{apply}{apply}} \code{(\LinkA{embed}{embed}(x,k),1,mean)} and other 
related functions.
\item \code{"keep"} - fill the ends with numbers from \code{x} vector 
\code{(out[1:k2] = x[1:k2])}
\item \code{"constant"} - fill the ends with first and last calculated 
value in output array \code{(out[1:k2] = out[k2+1])}
\item \code{"NA"} - fill the ends with NA's \code{(out[1:k2] = NA)}
\item \code{"func"} - same as \code{"mean"} but implimented
in R. This option could be very slow, and is included mostly for testing
}
Similar to \code{endrule} in \code{\LinkA{runmed}{runmed}} function which has the 
following options: \dQuote{\code{c("median", "keep", "constant")}} .

\end{ldescription}
\end{Arguments}
\begin{Details}\relax
Apart from the end values, the result of y = runmean(x, k) is the same as 
\dQuote{\code{for(j=(1+k2):(n-k2)) y[j]=mean(x[(j-k2):(j+k2)])}}.

The main incentive to write this set of functions was relative slowness of 
majority of moving window functions available in R and its packages.  With the 
exception of \code{\LinkA{runmed}{runmed}}, a running window median function, all 
functions listed in "see also" section are slower than very inefficient 
\dQuote{\code{\LinkA{apply}{apply}(\LinkA{apply}{apply}(x,k),1,FUN)}} approach. Relative 
speed of \code{runmean} function is O(n).

Function \code{EndRule} applies one of the five methods (see \code{endrule} 
argument) to process end-points of the input array \code{x}. In current 
version of the code the default \code{endrule="mean"} option is calculated 
within C code. That is done to improve speed in case of large moving windows.

In case of \code{runmean(..., alg="exact")} function a special algorithm is 
used (see references section) to ensure that round-off errors do not 
accumulate. As a result \code{runmean} is more accurate than 
\code{\LinkA{filter}{filter}}(x, rep(1/k,k)) and \code{runmean(..., alg="C")} 
functions.
\end{Details}
\begin{Value}
Returns a numeric vector of the same length as \code{x}. Only in case of 
\code{endrule="trim"} the output will be shorter.
\end{Value}
\begin{Note}\relax
Function \code{runmean(..., alg="exact")} is based by code by Vadim Ogranovich,
which is based on Python code (see last reference), pointed out by Gabor 
Grothendieck.
\end{Note}
\begin{Author}\relax
Jarek Tuszynski (SAIC) \email{jaroslaw.w.tuszynski@saic.com}
\end{Author}
\begin{References}\relax
\Itemize{       
\item About round-off error correction used in \code{runmean}:
Shewchuk, Jonathan \emph{Adaptive Precision Floating-Point Arithmetic and Fast 
Robust Geometric Predicates},  
\url{http://www-2.cs.cmu.edu/afs/cs/project/quake/public/papers/robust-arithmetic.ps}

\item More on round-off error correction can be found at:
\url{http://aspn.activestate.com/ASPN/Cookbook/Python/Recipe/393090 }  
}
\end{References}
\begin{SeeAlso}\relax
Links related to:
\Itemize{       
\item moving mean - \code{\LinkA{mean}{mean}}, \code{\LinkA{kernapply}{kernapply}}, 
\code{\LinkA{filter}{filter}}, \code{\LinkA{runsum.exact}{runsum.exact}}, \code{\LinkA{decompose}{decompose}},
\code{\LinkA{stl}{stl}},
\code{\LinkA{rollMean}{rollMean}} from \pkg{fSeries} library, 
\code{\LinkA{rollmean}{rollmean}} from \pkg{zoo} library,
\code{\LinkA{subsums}{subsums}} from \pkg{magic} library,
\item Other moving window functions  from this package: \code{\LinkA{runmin}{runmin}}, 
\code{\LinkA{runmax}{runmax}}, \code{\LinkA{runquantile}{runquantile}}, \code{\LinkA{runmad}{runmad}} and
\code{\LinkA{runsd}{runsd}} 
\item \code{\LinkA{runmed}{runmed}}
\item generic running window functions: \code{\LinkA{apply}{apply}}\code{
     (\LinkA{embed}{embed}(x,k), 1, FUN)} (fastest), \code{\LinkA{rollFun}{rollFun}} 
from \pkg{fSeries} (slow), \code{\LinkA{running}{running}} from \pkg{gtools} 
package (extremely slow for this purpose), \code{\LinkA{rapply}{rapply}} from 
\pkg{zoo} library, \code{\LinkA{subsums}{subsums}} from 
\pkg{magic} library can perform running window operations on data with any 
dimensions. 
}
\end{SeeAlso}
\begin{Examples}
\begin{ExampleCode}
  # show runmean for different window sizes
  n=200;
  x = rnorm(n,sd=30) + abs(seq(n)-n/4)
  x[seq(1,200,10)] = NaN;              # add NANs
  col = c("black", "red", "green", "blue", "magenta", "cyan")
  plot(x, col=col[1], main = "Moving Window Means")
  lines(runmean(x, 3), col=col[2])
  lines(runmean(x, 8), col=col[3])
  lines(runmean(x,15), col=col[4])
  lines(runmean(x,24), col=col[5])
  lines(runmean(x,50), col=col[6])
  lab = c("data", "k=3", "k=8", "k=15", "k=24", "k=50")
  legend(0,0.9*n, lab, col=col, lty=1 )
  
  # basic tests against 2 standard R approaches
  k=25; n=200;
  x = rnorm(n,sd=30) + abs(seq(n)-n/4)      # create random data
  a = runmean(x,k, endrule="trim")          # tested function
  b = apply(embed(x,k), 1, mean)            # approach #1
  c = cumsum(c( sum(x[1:k]), diff(x,k) ))/k # approach #2
  eps = .Machine$double.eps ^ 0.5
  stopifnot(all(abs(a-b)<eps));
  stopifnot(all(abs(a-c)<eps));
  
  # test against loop approach
  # this test works fine at the R prompt but fails during package check - need to investigate
  k=25; 
  data(iris)
  x = iris[,1]
  n = length(x)
  x[seq(1,n,11)] = NaN;                # add NANs
  k2 = k
  k1 = k-k2-1
  a = runmean(x, k)
  b = array(0,n)
  for(j in 1:n) {
    lo = max(1, j-k1)
    hi = min(n, j+k2)
    b[j] = mean(x[lo:hi], na.rm = TRUE)
  }
  #stopifnot(all(abs(a-b)<eps)); # commented out for time beeing - on to do list
  
  # compare calculation at array ends
  a = runmean(x, k, endrule="mean")  # fast C code
  b = runmean(x, k, endrule="func")  # slow R code
  stopifnot(all(abs(a-b)<eps));
  
  # Testing of different methods to each other for non-finite data
  # Only alg "C" and "exact" can handle not finite numbers 
  eps = .Machine$double.eps ^ 0.5
  n=200;  k=51;
  x = rnorm(n,sd=30) + abs(seq(n)-n/4) # nice behaving data
  x[seq(1,n,10)] = NaN;                # add NANs
  x[seq(1,n, 9)] = Inf;                # add infinities
  b = runmean( x, k, alg="C")
  c = runmean( x, k, alg="exact")
  stopifnot(all(abs(b-c)<eps));

  # Test if moving windows forward and backward gives the same results
  # Test also performed on data with non-finite numbers
  a = runmean(x     , alg="C", k)
  b = runmean(x[n:1], alg="C", k)
  stopifnot(all(abs(a[n:1]-b)<eps));
  a = runmean(x     , alg="exact", k)
  b = runmean(x[n:1], alg="exact", k)
  stopifnot(all(abs(a[n:1]-b)<eps));

  # Exhaustive testing of different methods to each other for different windows
  numeric.test = function (x, k) {
    a = runmean( x, k, alg="fast")
    b = runmean( x, k, alg="C")
    c = runmean( x, k, alg="exact")
    d = runmean( x, k, alg="R", endrule="func")
    eps = .Machine$double.eps ^ 0.5
    stopifnot(all(abs(a-b)<eps));
    stopifnot(all(abs(b-c)<eps));
    stopifnot(all(abs(c-d)<eps));
  }
  n=200;
  x = rnorm(n,sd=30) + abs(seq(n)-n/4) # nice behaving data
  for(i in 1:5) numeric.test(x, i)     # test small window sizes
  for(i in 1:5) numeric.test(x, n-i+1) # test large window size

  # speed comparison
  ## Not run: 
  x=runif(1e7); k=1e4;
  system.time(runmean(x,k,alg="fast"))
  system.time(runmean(x,k,alg="C"))
  system.time(runmean(x,k,alg="exact"))
  system.time(runmean(x,k,alg="R"))           # R version of the function
  x=runif(1e5); k=1e2;                        # reduce vector and window sizes
  system.time(runmean(x,k,alg="R"))           # R version of the function
  system.time(apply(embed(x,k), 1, mean))     # standard R approach
  system.time(filter(x, rep(1/k,k), sides=2)) # the fastest alternative I know 
  
## End(Not run)
   
  # show different runmean algorithms with data spanning many orders of magnitude
  n=30; k=5;
  x = rep(100/3,n)
  d=1e10
  x[5] = d;     
  x[13] = d; 
  x[14] = d*d; 
  x[15] = d*d*d; 
  x[16] = d*d*d*d; 
  x[17] = d*d*d*d*d; 
  a = runmean(x, k, alg="fast" )
  b = runmean(x, k, alg="C"    )
  c = runmean(x, k, alg="exact")
  y = t(rbind(x,a,b,c))
  y
\end{ExampleCode}
\end{Examples}

