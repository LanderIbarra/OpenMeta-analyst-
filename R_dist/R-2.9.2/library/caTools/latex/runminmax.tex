\HeaderA{runmin \& runmax}{Minimum and Maximum of Moving Windows}{runmin .Ramp. runmax}
\aliasA{runmax}{runmin \& runmax}{runmax}
\aliasA{runmin}{runmin \& runmax}{runmin}
\keyword{ts}{runmin \& runmax}
\keyword{smooth}{runmin \& runmax}
\keyword{array}{runmin \& runmax}
\keyword{utilities}{runmin \& runmax}
\begin{Description}\relax
Moving (aka running, rolling) Window Minimum and Maximum 
calculated over a vector
\end{Description}
\begin{Usage}
\begin{verbatim}
  runmin(x, k, alg=c("C", "R"),
         endrule=c("min", "NA", "trim", "keep", "constant", "func"))
  runmax(x, k, alg=c("C", "R"),
         endrule=c("max", "NA", "trim", "keep", "constant", "func"))
\end{verbatim}
\end{Usage}
\begin{Arguments}
\begin{ldescription}
\item[\code{x}] numeric vector of length n
\item[\code{k}] width of moving window; must be an integer between one and n 
\item[\code{endrule}] character string indicating how the values at the beginning 
and the end, of the array, should be treated. Only first and last \code{k2} 
values at both ends are affected, where \code{k2} is the half-bandwidth 
\code{k2 = k \%/\% 2}.
\Itemize{
\item \code{"min"} \& \code{"max"} - applies the underlying function to 
smaller and smaller sections of the array. In case of min equivalent to: 
\code{for(i in 1:k2) out[i]=min(x[1:(i+k2)])}. Default.
\item \code{"trim"} - trim the ends; output array length is equal to 
\code{length(x)-2*k2 (out = out[(k2+1):(n-k2)])}. This option mimics 
output of \code{\LinkA{apply}{apply}} \code{(\LinkA{embed}{embed}(x,k),1,FUN)} and other 
related functions.
\item \code{"keep"} - fill the ends with numbers from \code{x} vector 
\code{(out[1:k2] = x[1:k2])}
\item \code{"constant"} - fill the ends with first and last calculated 
value in output array \code{(out[1:k2] = out[k2+1])}
\item \code{"NA"} - fill the ends with NA's \code{(out[1:k2] = NA)}
\item \code{"func"} - same as \code{"min"} \& \code{"max"} but implimented
in R. This option could be very slow, and is included mostly for testing
}
Similar to \code{endrule} in \code{\LinkA{runmed}{runmed}} function which has the 
following options: \dQuote{\code{c("median", "keep", "constant")}} .

\item[\code{alg}] an option allowing to choose different algorithms or 
implementations. Default is to use of code written in C (option \code{alg="C"}).
Option \code{alg="R"} will use slower code written in R. Useful for 
debugging and studying the algorithm.
\end{ldescription}
\end{Arguments}
\begin{Details}\relax
Apart from the end values, the result of y = runFUN(x, k) is the same as 
\dQuote{\code{for(j=(1+k2):(n-k2)) y[j]=FUN(x[(j-k2):(j+k2)], na.rm = TRUE)}}, where FUN 
stands for min or max functions.  Both functions can handle non-finite 
numbers like NaN's and Inf's the same way as \code{\LinkA{min}{min}(x, na.rm = TRUE)}).


The main incentive to write this set of functions was relative slowness of 
majority of moving window functions available in R and its packages.  With the 
exception of \code{\LinkA{runmed}{runmed}}, a running window median function, all 
functions listed in "see also" section are slower than very inefficient 
\dQuote{\code{\LinkA{apply}{apply}(\LinkA{apply}{apply}(x,k),1,FUN)}} approach. Relative 
speeds \code{runmin} and \code{runmax} functions is O(n) in best and average 
case and O(n*k) in worst case.

Both functions work with infinite numbers (\code{NA},\code{NaN},\code{Inf},
\code{-Inf}). Also default \code{endrule} is hardwired in C for speed.
\end{Details}
\begin{Value}
Returns a numeric vector of the same length as \code{x}. Only in case of 
\code{endrule="trim"} the output will be shorter.
\end{Value}
\begin{Author}\relax
Jarek Tuszynski (SAIC) \email{jaroslaw.w.tuszynski@saic.com}
\end{Author}
\begin{SeeAlso}\relax
Links related to:
\Itemize{       
\item Other moving window functions  from this package: \code{\LinkA{runmean}{runmean}}, 
\code{\LinkA{runquantile}{runquantile}}, \code{\LinkA{runmad}{runmad}} and \code{\LinkA{runsd}{runsd}}  
\item R functions: \code{\LinkA{runmed}{runmed}}, \code{\LinkA{min}{min}}, \code{\LinkA{max}{max}}
\item Similar functions in other packages: \code{\LinkA{rollMin}{rollMin}} 
and \code{\LinkA{rollMax}{rollMax}} from \pkg{fSeries} library
\code{\LinkA{rollmax}{rollmax}} from \pkg{zoo} library
\item Generic running window functions: \code{\LinkA{apply}{apply}}\code{
     (\LinkA{embed}{embed}(x,k), 1, FUN)} (fastest), \code{\LinkA{rollFun}{rollFun}} 
from \pkg{fSeries} (slow), \code{\LinkA{running}{running}} from \pkg{gtools} 
package (extremely slow for this purpose), \code{\LinkA{rapply}{rapply}} from 
\pkg{zoo} library, \code{\LinkA{subsums}{subsums}} from 
\pkg{magic} library can perform running window operations on data with any 
dimensions. 
}
\end{SeeAlso}
\begin{Examples}
\begin{ExampleCode}
  # show plot using runmin, runmax and runmed
  k=25; n=200;
  x = rnorm(n,sd=30) + abs(seq(n)-n/4)
  col = c("black", "red", "green", "blue", "magenta", "cyan")
  plot(x, col=col[1], main = "Moving Window Analysis Functions")
  lines(runmin(x,k), col=col[2])
  lines(runmean(x,k), col=col[3])
  lines(runmax(x,k), col=col[4])
  legend(0,.9*n, c("data", "runmin", "runmean", "runmax"), col=col, lty=1 )

  # basic tests against standard R approach
  a = runmin(x,k, endrule="trim") # test only the inner part 
  b = apply(embed(x,k), 1, min)   # Standard R running min
  stopifnot(all(a==b));
  a = runmax(x,k, endrule="trim") # test only the inner part
  b = apply(embed(x,k), 1, max)   # Standard R running min
  stopifnot(all(a==b));
  
  # test against loop approach
  # this test works fine at the R prompt but fails during package check - need to investigate
  k=25; 
  data(iris)
  x = iris[,1]
  n = length(x)
  x[seq(1,n,11)] = NaN;                # add NANs
  k2 = k
  k1 = k-k2-1
  a1 = runmin(x, k)
  a2 = runmax(x, k)
  b1 = array(0,n)
  b2 = array(0,n)
  for(j in 1:n) {
    lo = max(1, j-k1)
    hi = min(n, j+k2)
    b1[j] = min(x[lo:hi], na.rm = TRUE)
    b2[j] = max(x[lo:hi], na.rm = TRUE)
  }
  #stopifnot(all(a1==b1), na.rm=TRUE);
  #stopifnot(all(a2==b2), na.rm=TRUE);
  
  # Test if moving windows forward and backward gives the same results
  # Two data sets also corespond to best and worst-case scenatio data-sets
  k=51; n=200;
  a = runmin(n:1, k) 
  b = runmin(1:n, k)
  stopifnot(all(a[n:1]==b, na.rm=TRUE));
  a = runmax(n:1, k)
  b = runmax(1:n, k)
  stopifnot(all(a[n:1]==b, na.rm=TRUE));

  # Compare C and R algorithms to each other for extreme window sizes
  numeric.test = function (x, k) {
    a = runmin( x, k, alg="C")
    b = runmin( x, k, alg="R")
    c =-runmax(-x, k, alg="C")
    d =-runmax(-x, k, alg="R")
    stopifnot(all(a==b, na.rm=TRUE));
    #stopifnot(all(c==d, na.rm=TRUE)); 
    #stopifnot(all(a==c, na.rm=TRUE));
    stopifnot(all(b==d, na.rm=TRUE));
  }
  n=200;                               # n is an even number
  x = rnorm(n,sd=30) + abs(seq(n)-n/4) # random data
  for(i in 1:5) numeric.test(x, i)     # test for small window size
  for(i in 1:5) numeric.test(x, n-i+1) # test for large window size
  n=201;                               # n is an odd number
  x = rnorm(n,sd=30) + abs(seq(n)-n/4) # random data
  for(i in 1:5) numeric.test(x, i)     # test for small window size
  for(i in 1:5) numeric.test(x, n-i+1) # test for large window size
  n=200;                               # n is an even number
  x = rnorm(n,sd=30) + abs(seq(n)-n/4) # random data
  x[seq(1,200,10)] = NaN;              # with some NaNs
  for(i in 1:5) numeric.test(x, i)     # test for small window size
  for(i in 1:5) numeric.test(x, n-i+1) # test for large window size
  n=201;                               # n is an odd number
  x = rnorm(n,sd=30) + abs(seq(n)-n/4) # random data
  x[seq(1,200,2)] = NaN;               # with some NaNs
  for(i in 1:5) numeric.test(x, i)     # test for small window size
  for(i in 1:5) numeric.test(x, n-i+1) # test for large window size

  # speed comparison
  ## Not run: 
  n = 1e7;  k=991; 
  x1 = runif(n);                       # random data - average case scenario
  x2 = 1:n;                            #  best-case scenario data for runmax
  x3 = n:1;                            # worst-case scenario data for runmax
  system.time( runmax( x1,k,alg="C"))  # C alg on average data O(n)
  system.time( runmax( x2,k,alg="C"))  # C alg on  best-case data O(n)
  system.time( runmax( x3,k,alg="C"))  # C alg on worst-case data O(n*k)
  system.time(-runmin(-x1,k,alg="C"))  # use runmin to do runmax work
  system.time( runmax( x1,k,alg="R"))  # R version of the function
  x=runif(1e5); k=1e2;                 # reduce vector and window sizes
  system.time(runmax(x,k,alg="R"))     # R version of the function
  system.time(apply(embed(x,k), 1, max)) # standard R approach 
  
## End(Not run)
\end{ExampleCode}
\end{Examples}

