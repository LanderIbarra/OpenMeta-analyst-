\HeaderA{jjplot}{The workhorse plotting function for the jjplot plotting library.}{jjplot}
\begin{Description}\relax
This is the main entry point for generating plots via the jjplot
library.   It is called by specifying a data frame on which
to operate, along with a formula which defines aesthetic mappings
along with a tree of statistic and geometry commands.
\end{Description}
\begin{Usage}
\begin{verbatim}
jjplot(f, data = NULL, color = NULL,
       fill = NULL, size = NULL, alpha = 1.0,
       log.x = FALSE, log.y = FALSE,
       xlab = NULL, ylab = NULL,
       expand = c(0.04, 0.04),
       color.scale = NULL)
\end{verbatim}
\end{Usage}
\begin{Arguments}
\begin{ldescription}
\item[\code{f}] A formula specifying mappings for x and y aesthetics, as well as the
statistic and geometry commands.  See details. 

\item[\code{data}] A data frame in whose scope the data expressions (i.e., \code{f})
should be evaluated. 

\item[\code{color}] The expression to map to the color aesthetic.

\item[\code{fill}] The expression to map to the fill aesthetic.

\item[\code{size}] The expression to map to the size aesthetic.

\item[\code{alpha}] A constant numeric value with which to specify the alpha blending
used on color and fill scales.

\item[\code{log.x}] Use a logarithmic x scale.

\item[\code{log.y}] Use a logarithmic y scale.

\item[\code{xlab}] When specified, this overrides the default label for the x axis.

\item[\code{ylab}] When specified, this overrides the default label for the y axis.

\item[\code{expand}] A length-2 numeric vector.  The axes are guaranteed to
contain the range of the data plus a fraction of the data range
specified by this parameter.  The default simulates the default
behavior of R's base plotting functions.  
\item[\code{color.scale}] A function which maps values between 0 and 1 to colors.  If
specified this color scale will override the default automatically
generated color scale.

\end{ldescription}
\end{Arguments}
\begin{Details}\relax
The formula is specified by placing the optional y aesthetic to the
left of the '\textasciitilde{}' and the x aesthetic as 
the first term (rightmost) of the formula.  Layers are subsequently
evaluated from right to left, separated by '+'.  Parentheses are
grouping operators for layers.  Statistics can be applied using the
':' operator, with the statistic on the right hand side and operations
depending on the statistic on the left hand side.

Full up-to-date documentation with examples can be found at:
\url{http://code.google.com/p/jjplot/wiki/Documentation}
\end{Details}
\begin{Value}
The function is called only for its side effect (namely, a plot!).
\end{Value}
\begin{Author}\relax
Jonathan Chang \textless{}jcone@princeton.edu\textgreater{}
Eytan Bakshy \textless{}ebakshy@gmail.com\textgreater{}
\end{Author}
\begin{References}\relax
\url{http://code.google.com/p/jjplot/wiki/Documentation}
\end{References}
\begin{Examples}
\begin{ExampleCode}
 ## See the demos for example usages.
 ## Not run: demo(jjplot)

 ## See the following demo for speed tests.
 ## Not run: demo(jjplot.performance)
\end{ExampleCode}
\end{Examples}

