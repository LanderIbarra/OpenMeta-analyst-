\HeaderA{jjplot-package}{The jjplot plotting library for making beautiful plots fast and simple.}{jjplot.Rdash.package}
\keyword{package}{jjplot-package}
\begin{Description}\relax
jjplot is a plotting library inspired by ggplot2.  jjplot is written in
grid and while not yet as functional as ggplot2, jjplot can quickly and
easily produce beautiful graphics including line graphs, scatter plots,
bar graphs, and boxplots.  jjplot also makes it straightfoward to
compute and overlay statistics and fits onto graphs, and to facet graphs
by factor via subplots and automatic color scales.
\end{Description}
\begin{Details}\relax
\Tabular{ll}{
Package: & jjplot\\
Type: & Package\\
Version: & 1.0\\
Date: & 2010-02-24\\
License: & LGPL\\
LazyLoad: & yes\\
}

Full up-to-date documentation with examples can be found at:
\url{http://code.google.com/p/jjplot/wiki/Documentation}
\end{Details}
\begin{Author}\relax
Jonathan Chang \textless{}jcone@princeton.edu\textgreater{}
Eytan Bakshy \textless{}ebakshy@gmail.com\textgreater{}
\end{Author}
\begin{References}\relax
The \code{ggplot2} package is the inpsiration for (and a more
feature-rich version of) this package.
\end{References}
\begin{SeeAlso}\relax
\code{\LinkA{jjplot}{jjplot}}
\end{SeeAlso}
\begin{Examples}
\begin{ExampleCode}
 ## See the demos for example usages.
 ## Not run: demo(jjplot)

 ## See the following demo for speed tests.
 ## Not run: demo(jjplot.performance)
\end{ExampleCode}
\end{Examples}

